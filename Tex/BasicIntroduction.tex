
% Default to the notebook output style

    


% Inherit from the specified cell style.




    
\documentclass[11pt]{article}

    
    
    \usepackage[T1]{fontenc}
    % Nicer default font than Computer Modern for most use cases
    \usepackage{palatino}

    % Basic figure setup, for now with no caption control since it's done
    % automatically by Pandoc (which extracts ![](path) syntax from Markdown).
    \usepackage{graphicx}
    % We will generate all images so they have a width \maxwidth. This means
    % that they will get their normal width if they fit onto the page, but
    % are scaled down if they would overflow the margins.
    \makeatletter
    \def\maxwidth{\ifdim\Gin@nat@width>\linewidth\linewidth
    \else\Gin@nat@width\fi}
    \makeatother
    \let\Oldincludegraphics\includegraphics
    % Set max figure width to be 80% of text width, for now hardcoded.
    \renewcommand{\includegraphics}[1]{\Oldincludegraphics[width=.8\maxwidth]{#1}}
    % Ensure that by default, figures have no caption (until we provide a
    % proper Figure object with a Caption API and a way to capture that
    % in the conversion process - todo).
    \usepackage{caption}
    \DeclareCaptionLabelFormat{nolabel}{}
    \captionsetup{labelformat=nolabel}

    \usepackage{adjustbox} % Used to constrain images to a maximum size 
    \usepackage{xcolor} % Allow colors to be defined
    \usepackage{enumerate} % Needed for markdown enumerations to work
    \usepackage{geometry} % Used to adjust the document margins
    \usepackage{amsmath} % Equations
    \usepackage{amssymb} % Equations
    \usepackage{textcomp} % defines textquotesingle
    % Hack from http://tex.stackexchange.com/a/47451/13684:
    \AtBeginDocument{%
        \def\PYZsq{\textquotesingle}% Upright quotes in Pygmentized code
    }
    \usepackage{upquote} % Upright quotes for verbatim code
    \usepackage{eurosym} % defines \euro
    \usepackage[mathletters]{ucs} % Extended unicode (utf-8) support
    \usepackage[utf8x]{inputenc} % Allow utf-8 characters in the tex document
    \usepackage{fancyvrb} % verbatim replacement that allows latex
    \usepackage{grffile} % extends the file name processing of package graphics 
                         % to support a larger range 
    % The hyperref package gives us a pdf with properly built
    % internal navigation ('pdf bookmarks' for the table of contents,
    % internal cross-reference links, web links for URLs, etc.)
    \usepackage{hyperref}
    \usepackage{longtable} % longtable support required by pandoc >1.10
    \usepackage{booktabs}  % table support for pandoc > 1.12.2
    \usepackage[normalem]{ulem} % ulem is needed to support strikethroughs (\sout)
                                % normalem makes italics be italics, not underlines
    

    
    
    % Colors for the hyperref package
    \definecolor{urlcolor}{rgb}{0,.145,.698}
    \definecolor{linkcolor}{rgb}{.71,0.21,0.01}
    \definecolor{citecolor}{rgb}{.12,.54,.11}

    % ANSI colors
    \definecolor{ansi-black}{HTML}{3E424D}
    \definecolor{ansi-black-intense}{HTML}{282C36}
    \definecolor{ansi-red}{HTML}{E75C58}
    \definecolor{ansi-red-intense}{HTML}{B22B31}
    \definecolor{ansi-green}{HTML}{00A250}
    \definecolor{ansi-green-intense}{HTML}{007427}
    \definecolor{ansi-yellow}{HTML}{DDB62B}
    \definecolor{ansi-yellow-intense}{HTML}{B27D12}
    \definecolor{ansi-blue}{HTML}{208FFB}
    \definecolor{ansi-blue-intense}{HTML}{0065CA}
    \definecolor{ansi-magenta}{HTML}{D160C4}
    \definecolor{ansi-magenta-intense}{HTML}{A03196}
    \definecolor{ansi-cyan}{HTML}{60C6C8}
    \definecolor{ansi-cyan-intense}{HTML}{258F8F}
    \definecolor{ansi-white}{HTML}{C5C1B4}
    \definecolor{ansi-white-intense}{HTML}{A1A6B2}

    % commands and environments needed by pandoc snippets
    % extracted from the output of `pandoc -s`
    \providecommand{\tightlist}{%
      \setlength{\itemsep}{0pt}\setlength{\parskip}{0pt}}
    \DefineVerbatimEnvironment{Highlighting}{Verbatim}{commandchars=\\\{\}}
    % Add ',fontsize=\small' for more characters per line
    \newenvironment{Shaded}{}{}
    \newcommand{\KeywordTok}[1]{\textcolor[rgb]{0.00,0.44,0.13}{\textbf{{#1}}}}
    \newcommand{\DataTypeTok}[1]{\textcolor[rgb]{0.56,0.13,0.00}{{#1}}}
    \newcommand{\DecValTok}[1]{\textcolor[rgb]{0.25,0.63,0.44}{{#1}}}
    \newcommand{\BaseNTok}[1]{\textcolor[rgb]{0.25,0.63,0.44}{{#1}}}
    \newcommand{\FloatTok}[1]{\textcolor[rgb]{0.25,0.63,0.44}{{#1}}}
    \newcommand{\CharTok}[1]{\textcolor[rgb]{0.25,0.44,0.63}{{#1}}}
    \newcommand{\StringTok}[1]{\textcolor[rgb]{0.25,0.44,0.63}{{#1}}}
    \newcommand{\CommentTok}[1]{\textcolor[rgb]{0.38,0.63,0.69}{\textit{{#1}}}}
    \newcommand{\OtherTok}[1]{\textcolor[rgb]{0.00,0.44,0.13}{{#1}}}
    \newcommand{\AlertTok}[1]{\textcolor[rgb]{1.00,0.00,0.00}{\textbf{{#1}}}}
    \newcommand{\FunctionTok}[1]{\textcolor[rgb]{0.02,0.16,0.49}{{#1}}}
    \newcommand{\RegionMarkerTok}[1]{{#1}}
    \newcommand{\ErrorTok}[1]{\textcolor[rgb]{1.00,0.00,0.00}{\textbf{{#1}}}}
    \newcommand{\NormalTok}[1]{{#1}}
    
    % Additional commands for more recent versions of Pandoc
    \newcommand{\ConstantTok}[1]{\textcolor[rgb]{0.53,0.00,0.00}{{#1}}}
    \newcommand{\SpecialCharTok}[1]{\textcolor[rgb]{0.25,0.44,0.63}{{#1}}}
    \newcommand{\VerbatimStringTok}[1]{\textcolor[rgb]{0.25,0.44,0.63}{{#1}}}
    \newcommand{\SpecialStringTok}[1]{\textcolor[rgb]{0.73,0.40,0.53}{{#1}}}
    \newcommand{\ImportTok}[1]{{#1}}
    \newcommand{\DocumentationTok}[1]{\textcolor[rgb]{0.73,0.13,0.13}{\textit{{#1}}}}
    \newcommand{\AnnotationTok}[1]{\textcolor[rgb]{0.38,0.63,0.69}{\textbf{\textit{{#1}}}}}
    \newcommand{\CommentVarTok}[1]{\textcolor[rgb]{0.38,0.63,0.69}{\textbf{\textit{{#1}}}}}
    \newcommand{\VariableTok}[1]{\textcolor[rgb]{0.10,0.09,0.49}{{#1}}}
    \newcommand{\ControlFlowTok}[1]{\textcolor[rgb]{0.00,0.44,0.13}{\textbf{{#1}}}}
    \newcommand{\OperatorTok}[1]{\textcolor[rgb]{0.40,0.40,0.40}{{#1}}}
    \newcommand{\BuiltInTok}[1]{{#1}}
    \newcommand{\ExtensionTok}[1]{{#1}}
    \newcommand{\PreprocessorTok}[1]{\textcolor[rgb]{0.74,0.48,0.00}{{#1}}}
    \newcommand{\AttributeTok}[1]{\textcolor[rgb]{0.49,0.56,0.16}{{#1}}}
    \newcommand{\InformationTok}[1]{\textcolor[rgb]{0.38,0.63,0.69}{\textbf{\textit{{#1}}}}}
    \newcommand{\WarningTok}[1]{\textcolor[rgb]{0.38,0.63,0.69}{\textbf{\textit{{#1}}}}}
    
    
    % Define a nice break command that doesn't care if a line doesn't already
    % exist.
    \def\br{\hspace*{\fill} \\* }
    % Math Jax compatability definitions
    \def\gt{>}
    \def\lt{<}
    % Document parameters
    \title{BasicIntroduction}
    
    
    

    % Pygments definitions
    
\makeatletter
\def\PY@reset{\let\PY@it=\relax \let\PY@bf=\relax%
    \let\PY@ul=\relax \let\PY@tc=\relax%
    \let\PY@bc=\relax \let\PY@ff=\relax}
\def\PY@tok#1{\csname PY@tok@#1\endcsname}
\def\PY@toks#1+{\ifx\relax#1\empty\else%
    \PY@tok{#1}\expandafter\PY@toks\fi}
\def\PY@do#1{\PY@bc{\PY@tc{\PY@ul{%
    \PY@it{\PY@bf{\PY@ff{#1}}}}}}}
\def\PY#1#2{\PY@reset\PY@toks#1+\relax+\PY@do{#2}}

\def\PY@tok@gd{\def\PY@tc##1{\textcolor[rgb]{0.63,0.00,0.00}{##1}}}
\def\PY@tok@gu{\let\PY@bf=\textbf\def\PY@tc##1{\textcolor[rgb]{0.50,0.00,0.50}{##1}}}
\def\PY@tok@gt{\def\PY@tc##1{\textcolor[rgb]{0.00,0.25,0.82}{##1}}}
\def\PY@tok@gs{\let\PY@bf=\textbf}
\def\PY@tok@gr{\def\PY@tc##1{\textcolor[rgb]{1.00,0.00,0.00}{##1}}}
\def\PY@tok@cm{\let\PY@it=\textit\def\PY@tc##1{\textcolor[rgb]{0.25,0.50,0.50}{##1}}}
\def\PY@tok@vg{\def\PY@tc##1{\textcolor[rgb]{0.10,0.09,0.49}{##1}}}
\def\PY@tok@m{\def\PY@tc##1{\textcolor[rgb]{0.40,0.40,0.40}{##1}}}
\def\PY@tok@mh{\def\PY@tc##1{\textcolor[rgb]{0.40,0.40,0.40}{##1}}}
\def\PY@tok@go{\def\PY@tc##1{\textcolor[rgb]{0.50,0.50,0.50}{##1}}}
\def\PY@tok@ge{\let\PY@it=\textit}
\def\PY@tok@vc{\def\PY@tc##1{\textcolor[rgb]{0.10,0.09,0.49}{##1}}}
\def\PY@tok@il{\def\PY@tc##1{\textcolor[rgb]{0.40,0.40,0.40}{##1}}}
\def\PY@tok@cs{\let\PY@it=\textit\def\PY@tc##1{\textcolor[rgb]{0.25,0.50,0.50}{##1}}}
\def\PY@tok@cp{\def\PY@tc##1{\textcolor[rgb]{0.74,0.48,0.00}{##1}}}
\def\PY@tok@gi{\def\PY@tc##1{\textcolor[rgb]{0.00,0.63,0.00}{##1}}}
\def\PY@tok@gh{\let\PY@bf=\textbf\def\PY@tc##1{\textcolor[rgb]{0.00,0.00,0.50}{##1}}}
\def\PY@tok@ni{\let\PY@bf=\textbf\def\PY@tc##1{\textcolor[rgb]{0.60,0.60,0.60}{##1}}}
\def\PY@tok@nl{\def\PY@tc##1{\textcolor[rgb]{0.63,0.63,0.00}{##1}}}
\def\PY@tok@nn{\let\PY@bf=\textbf\def\PY@tc##1{\textcolor[rgb]{0.00,0.00,1.00}{##1}}}
\def\PY@tok@no{\def\PY@tc##1{\textcolor[rgb]{0.53,0.00,0.00}{##1}}}
\def\PY@tok@na{\def\PY@tc##1{\textcolor[rgb]{0.49,0.56,0.16}{##1}}}
\def\PY@tok@nb{\def\PY@tc##1{\textcolor[rgb]{0.00,0.50,0.00}{##1}}}
\def\PY@tok@nc{\let\PY@bf=\textbf\def\PY@tc##1{\textcolor[rgb]{0.00,0.00,1.00}{##1}}}
\def\PY@tok@nd{\def\PY@tc##1{\textcolor[rgb]{0.67,0.13,1.00}{##1}}}
\def\PY@tok@ne{\let\PY@bf=\textbf\def\PY@tc##1{\textcolor[rgb]{0.82,0.25,0.23}{##1}}}
\def\PY@tok@nf{\def\PY@tc##1{\textcolor[rgb]{0.00,0.00,1.00}{##1}}}
\def\PY@tok@si{\let\PY@bf=\textbf\def\PY@tc##1{\textcolor[rgb]{0.73,0.40,0.53}{##1}}}
\def\PY@tok@s2{\def\PY@tc##1{\textcolor[rgb]{0.73,0.13,0.13}{##1}}}
\def\PY@tok@vi{\def\PY@tc##1{\textcolor[rgb]{0.10,0.09,0.49}{##1}}}
\def\PY@tok@nt{\let\PY@bf=\textbf\def\PY@tc##1{\textcolor[rgb]{0.00,0.50,0.00}{##1}}}
\def\PY@tok@nv{\def\PY@tc##1{\textcolor[rgb]{0.10,0.09,0.49}{##1}}}
\def\PY@tok@s1{\def\PY@tc##1{\textcolor[rgb]{0.73,0.13,0.13}{##1}}}
\def\PY@tok@sh{\def\PY@tc##1{\textcolor[rgb]{0.73,0.13,0.13}{##1}}}
\def\PY@tok@sc{\def\PY@tc##1{\textcolor[rgb]{0.73,0.13,0.13}{##1}}}
\def\PY@tok@sx{\def\PY@tc##1{\textcolor[rgb]{0.00,0.50,0.00}{##1}}}
\def\PY@tok@bp{\def\PY@tc##1{\textcolor[rgb]{0.00,0.50,0.00}{##1}}}
\def\PY@tok@c1{\let\PY@it=\textit\def\PY@tc##1{\textcolor[rgb]{0.25,0.50,0.50}{##1}}}
\def\PY@tok@kc{\let\PY@bf=\textbf\def\PY@tc##1{\textcolor[rgb]{0.00,0.50,0.00}{##1}}}
\def\PY@tok@c{\let\PY@it=\textit\def\PY@tc##1{\textcolor[rgb]{0.25,0.50,0.50}{##1}}}
\def\PY@tok@mf{\def\PY@tc##1{\textcolor[rgb]{0.40,0.40,0.40}{##1}}}
\def\PY@tok@err{\def\PY@bc##1{\fcolorbox[rgb]{1.00,0.00,0.00}{1,1,1}{##1}}}
\def\PY@tok@kd{\let\PY@bf=\textbf\def\PY@tc##1{\textcolor[rgb]{0.00,0.50,0.00}{##1}}}
\def\PY@tok@ss{\def\PY@tc##1{\textcolor[rgb]{0.10,0.09,0.49}{##1}}}
\def\PY@tok@sr{\def\PY@tc##1{\textcolor[rgb]{0.73,0.40,0.53}{##1}}}
\def\PY@tok@mo{\def\PY@tc##1{\textcolor[rgb]{0.40,0.40,0.40}{##1}}}
\def\PY@tok@kn{\let\PY@bf=\textbf\def\PY@tc##1{\textcolor[rgb]{0.00,0.50,0.00}{##1}}}
\def\PY@tok@mi{\def\PY@tc##1{\textcolor[rgb]{0.40,0.40,0.40}{##1}}}
\def\PY@tok@gp{\let\PY@bf=\textbf\def\PY@tc##1{\textcolor[rgb]{0.00,0.00,0.50}{##1}}}
\def\PY@tok@o{\def\PY@tc##1{\textcolor[rgb]{0.40,0.40,0.40}{##1}}}
\def\PY@tok@kr{\let\PY@bf=\textbf\def\PY@tc##1{\textcolor[rgb]{0.00,0.50,0.00}{##1}}}
\def\PY@tok@s{\def\PY@tc##1{\textcolor[rgb]{0.73,0.13,0.13}{##1}}}
\def\PY@tok@kp{\def\PY@tc##1{\textcolor[rgb]{0.00,0.50,0.00}{##1}}}
\def\PY@tok@w{\def\PY@tc##1{\textcolor[rgb]{0.73,0.73,0.73}{##1}}}
\def\PY@tok@kt{\def\PY@tc##1{\textcolor[rgb]{0.69,0.00,0.25}{##1}}}
\def\PY@tok@ow{\let\PY@bf=\textbf\def\PY@tc##1{\textcolor[rgb]{0.67,0.13,1.00}{##1}}}
\def\PY@tok@sb{\def\PY@tc##1{\textcolor[rgb]{0.73,0.13,0.13}{##1}}}
\def\PY@tok@k{\let\PY@bf=\textbf\def\PY@tc##1{\textcolor[rgb]{0.00,0.50,0.00}{##1}}}
\def\PY@tok@se{\let\PY@bf=\textbf\def\PY@tc##1{\textcolor[rgb]{0.73,0.40,0.13}{##1}}}
\def\PY@tok@sd{\let\PY@it=\textit\def\PY@tc##1{\textcolor[rgb]{0.73,0.13,0.13}{##1}}}

\def\PYZbs{\char`\\}
\def\PYZus{\char`\_}
\def\PYZob{\char`\{}
\def\PYZcb{\char`\}}
\def\PYZca{\char`\^}
\def\PYZsh{\char`\#}
\def\PYZpc{\char`\%}
\def\PYZdl{\char`\$}
\def\PYZti{\char`\~}
% for compatibility with earlier versions
\def\PYZat{@}
\def\PYZlb{[}
\def\PYZrb{]}
\makeatother


    % Exact colors from NB
    \definecolor{incolor}{rgb}{0.0, 0.0, 0.5}
    \definecolor{outcolor}{rgb}{0.545, 0.0, 0.0}



    
    % Prevent overflowing lines due to hard-to-break entities
    \sloppy 
    % Setup hyperref package
    \hypersetup{
      breaklinks=true,  % so long urls are correctly broken across lines
      colorlinks=true,
      urlcolor=urlcolor,
      linkcolor=linkcolor,
      citecolor=citecolor,
      }
    % Slightly bigger margins than the latex defaults
    
    \geometry{verbose,tmargin=1in,bmargin=1in,lmargin=1in,rmargin=1in}
    
    

    \begin{document}
    
    
    \maketitle
    
    

    
    \subsection{An Introduction to Basic
Julia}\label{an-introduction-to-basic-julia}

This quick introduction assumes that you have basic knowledge of some
scripting language and provides an example of the Julia syntax. So
before we explain anything, let's just treat it like a scripting
language, take a head-first dive into Julia, and see what happens.

You'll notice that, given the right syntax, almost everything will
``just work''. There will be some peculiarities, and these we will be
the facts which we will study in much more depth. Usually, these
oddies/differences from other scripting languages are ``the source of
Julia's power''.

    \subsubsection{Array Syntax}\label{array-syntax}

The array syntax is similar to MATLAB's conventions.

    \begin{Verbatim}[commandchars=\\\{\}]
{\color{incolor}In [{\color{incolor}11}]:} a = Vector\PYZob{}Float64\PYZcb{}(5) \PYZsh{} Create a length 5 Vector (dimension 1 array) of Float64's
         
         a = [1;2;3;4;5] \PYZsh{} Create the column vector [1 2 3 4 5]
         
         a = [1 2 3 4] \PYZsh{} Create the row vector [1 2 3 4]
         
         a[3] = 2 \PYZsh{} Change the third element of a (using linear indexing) to 2
         
         b = Matrix\PYZob{}Float64\PYZcb{}(4,2) \PYZsh{} Define a Matrix of Float64's of size (4,2)
         
         c = Array\PYZob{}Float64,4\PYZcb{}((4,5,6,7)) \PYZsh{} Define a (4,5,6,7) array of Float64's 
         
         mat    = [1 2 3 4
                   3 4 5 6
                   4 4 4 6
                   3 3 3 3] \PYZsh{}Define the matrix inline 
         
         mat[1,2] = 4 \PYZsh{} Set element (1,2) (row 1, column 2) to 4
         
         mat
\end{Verbatim}

            \begin{Verbatim}[commandchars=\\\{\}]
{\color{outcolor}Out[{\color{outcolor}11}]:} 4×4 Array\{Int64,2\}:
          1  4  3  4
          3  4  5  6
          4  4  4  6
          3  3  3  3
\end{Verbatim}
        
    Note that, in the console (called the REPL), you can use \texttt{;} to
surpress the output. In a script this is done automatically. Note that
the ``value'' of an array is its pointer to the memory location. This
means that arrays which are set equal affect the same values:

    \begin{Verbatim}[commandchars=\\\{\}]
{\color{incolor}In [{\color{incolor}12}]:} a = [1;3;4]
         b = a
         b[1] = 10
         a
\end{Verbatim}

            \begin{Verbatim}[commandchars=\\\{\}]
{\color{outcolor}Out[{\color{outcolor}12}]:} 3-element Array\{Int64,1\}:
          10
           3
           4
\end{Verbatim}
        
    To set an array equal to the values to another array, use copy

    \begin{Verbatim}[commandchars=\\\{\}]
{\color{incolor}In [{\color{incolor}13}]:} a = [1;4;5]
         b = copy(a)
         b[1] = 10
         a
\end{Verbatim}

            \begin{Verbatim}[commandchars=\\\{\}]
{\color{outcolor}Out[{\color{outcolor}13}]:} 3-element Array\{Int64,1\}:
          1
          4
          5
\end{Verbatim}
        
    We can also make an array of a similar size and shape via the function
\texttt{similar}, or make an array of zeros/ones with \texttt{zeros} or
\texttt{ones} respectively:

    \begin{Verbatim}[commandchars=\\\{\}]
{\color{incolor}In [{\color{incolor}14}]:} c = similar(a)
         d = zeros(a)
         e = ones(a)
         println(c); println(d); println(e)
\end{Verbatim}

    \begin{Verbatim}[commandchars=\\\{\}]
[786432,0,0]
[0,0,0]
[1,1,1]

    \end{Verbatim}

    Note that arrays can be index'd by arrays:

    \begin{Verbatim}[commandchars=\\\{\}]
{\color{incolor}In [{\color{incolor}15}]:} a[1:2]
\end{Verbatim}

            \begin{Verbatim}[commandchars=\\\{\}]
{\color{outcolor}Out[{\color{outcolor}15}]:} 2-element Array\{Int64,1\}:
          1
          4
\end{Verbatim}
        
    Arrays can be of any type, specified by the type parameter. One
interesting thing is that this means that arrays can be of arrays:

    \begin{Verbatim}[commandchars=\\\{\}]
{\color{incolor}In [{\color{incolor}12}]:} a = Vector\PYZob{}Vector\PYZob{}Float64\PYZcb{}\PYZcb{}(3)
         a[1] = [1;2;3]
         a[2] = [1;2]
         a[3] = [3;4;5]
         a
\end{Verbatim}

            \begin{Verbatim}[commandchars=\\\{\}]
{\color{outcolor}Out[{\color{outcolor}12}]:} 3-element Array\{Array\{Float64,1\},1\}:
          [1.0,2.0,3.0]
          [1.0,2.0]    
          [3.0,4.0,5.0]
\end{Verbatim}
        
    \begin{center}\rule{3in}{0.4pt}\end{center}

\paragraph{Question 1}\label{question-1}

Can you explain the following behavior? Julia's community values
consistancy of the rules, so all of the behavior is deducible from
simple rules. (Hint: I have noted all of the rules involved here).

    \begin{Verbatim}[commandchars=\\\{\}]
{\color{incolor}In [{\color{incolor}14}]:} b = a
         b[1] = [1;4;5]
         a
\end{Verbatim}

            \begin{Verbatim}[commandchars=\\\{\}]
{\color{outcolor}Out[{\color{outcolor}14}]:} 3-element Array\{Array\{Float64,1\},1\}:
          [1.0,4.0,5.0]
          [1.0,2.0]    
          [3.0,4.0,5.0]
\end{Verbatim}
        
    \begin{center}\rule{3in}{0.4pt}\end{center}

To fix this, there is a recursive copy function: \texttt{deepcopy}

    \begin{Verbatim}[commandchars=\\\{\}]
{\color{incolor}In [{\color{incolor}16}]:} b = deepcopy(a)
         b[1] = [1;2;3]
         a
\end{Verbatim}

            \begin{Verbatim}[commandchars=\\\{\}]
{\color{outcolor}Out[{\color{outcolor}16}]:} 3-element Array\{Array\{Float64,1\},1\}:
          [1.0,4.0,5.0]
          [1.0,2.0]    
          [3.0,4.0,5.0]
\end{Verbatim}
        
    For high performance, Julia provides mutating functions. These functions
change the input values that are passed in, instead of returning a new
value. By convention, mutating functions tend to be defined with a
\texttt{!} at the end and tend to mutate their first argument. An
example of a mutating function in \texttt{scale!} which scales an array
by a scalar (or array)

    \begin{Verbatim}[commandchars=\\\{\}]
{\color{incolor}In [{\color{incolor}19}]:} a = [1;6;8]
         scale!(a,2) \PYZsh{} a changes
\end{Verbatim}

            \begin{Verbatim}[commandchars=\\\{\}]
{\color{outcolor}Out[{\color{outcolor}19}]:} 3-element Array\{Int64,1\}:
           2
          12
          16
\end{Verbatim}
        
    The purpose of mutating functions is that they allow one to reduce the
number of memory allocations which is crucial for achiving high
performance.

    \subsection{Control Flow}\label{control-flow}

Control flow in Julia is pretty standard. You have your basic for and
while loops, and your if statements. There's more in the documentation.

    \begin{Verbatim}[commandchars=\\\{\}]
{\color{incolor}In [{\color{incolor}16}]:} for i=1:5 \PYZsh{}for i goes from 1 to 5
             println(i)
         end
         
         t = 0
         while t<5
             println(t)
             t+=1 \PYZsh{} t = t + 1
         end
         
         school = :UCI
         
         if school==:UCI
             println("ZotZotZot")
         else
             println("Not even worth discussing.")
         end
\end{Verbatim}

    \begin{Verbatim}[commandchars=\\\{\}]
1
2
3
4
5
0
1
2
3
4
ZotZotZot

    \end{Verbatim}

    One interesting feature about Julia control flow is that we can write
multiple loops in one line:

    \begin{Verbatim}[commandchars=\\\{\}]
{\color{incolor}In [{\color{incolor}17}]:} for i=1:2,j=2:4
             println(i*j)
         end
\end{Verbatim}

    \begin{Verbatim}[commandchars=\\\{\}]
2
3
4
4
6
8

    \end{Verbatim}

    \subsection{Function Syntax}\label{function-syntax}

    \begin{Verbatim}[commandchars=\\\{\}]
{\color{incolor}In [{\color{incolor}20}]:} f(x,y) = 2x+y \PYZsh{} Create an inline function
\end{Verbatim}

            \begin{Verbatim}[commandchars=\\\{\}]
{\color{outcolor}Out[{\color{outcolor}20}]:} f (generic function with 1 method)
\end{Verbatim}
        
    \begin{Verbatim}[commandchars=\\\{\}]
{\color{incolor}In [{\color{incolor}22}]:} f(1,2) \PYZsh{} Call the function
\end{Verbatim}

            \begin{Verbatim}[commandchars=\\\{\}]
{\color{outcolor}Out[{\color{outcolor}22}]:} 4
\end{Verbatim}
        
    \begin{Verbatim}[commandchars=\\\{\}]
{\color{incolor}In [{\color{incolor}24}]:} function f(x)
           x+2  
         end \PYZsh{} Long form definition
\end{Verbatim}

    \begin{Verbatim}[commandchars=\\\{\}]
WARNING: Method definition f(Any) in module Main at In[23]:2 overwritten at In[24]:2.

    \end{Verbatim}

            \begin{Verbatim}[commandchars=\\\{\}]
{\color{outcolor}Out[{\color{outcolor}24}]:} f (generic function with 2 methods)
\end{Verbatim}
        
    By default, Julia functions return the last value computed within them.

    \begin{Verbatim}[commandchars=\\\{\}]
{\color{incolor}In [{\color{incolor}26}]:} f(2)
\end{Verbatim}

            \begin{Verbatim}[commandchars=\\\{\}]
{\color{outcolor}Out[{\color{outcolor}26}]:} 4
\end{Verbatim}
        
    A key feature of Julia is multiple dispatch. Notice here that there is
``one function'', \texttt{f}, with two methods. Methods are the
actionable parts of a function. Here, there is one method defined as
\texttt{(::Any,::Any)} and \texttt{(::Any)}, meaning that if you give
\texttt{f} two values then it will call the first method, and if you
give it one value then it will call the second method.

Multiple dispatch works on types. To define a dispatch on a type, use
the \texttt{::Type} signifier:

    \begin{Verbatim}[commandchars=\\\{\}]
{\color{incolor}In [{\color{incolor}35}]:} f(x::Int,y::Int) = 3x+2y
\end{Verbatim}

    \begin{Verbatim}[commandchars=\\\{\}]
WARNING: Method definition f(Int64, Int64) in module Main at In[28]:1 overwritten at In[35]:1.

    \end{Verbatim}

            \begin{Verbatim}[commandchars=\\\{\}]
{\color{outcolor}Out[{\color{outcolor}35}]:} f (generic function with 4 methods)
\end{Verbatim}
        
    Julia will dispatch onto the strictest acceptible type signature.

    \begin{Verbatim}[commandchars=\\\{\}]
{\color{incolor}In [{\color{incolor}30}]:} f(2,3) \PYZsh{} 3x+2y
\end{Verbatim}

            \begin{Verbatim}[commandchars=\\\{\}]
{\color{outcolor}Out[{\color{outcolor}30}]:} 12
\end{Verbatim}
        
    \begin{Verbatim}[commandchars=\\\{\}]
{\color{incolor}In [{\color{incolor}32}]:} f(2.0,3) \PYZsh{} 2x+y since 2.0 is not an Int
\end{Verbatim}

            \begin{Verbatim}[commandchars=\\\{\}]
{\color{outcolor}Out[{\color{outcolor}32}]:} 7.0
\end{Verbatim}
        
    Types in signatures can be parametric. For example, we can define a
method for ``two values are passed in, both Numbers and having the same
type''. Note that \texttt{\textless{}:} means ``a subtype of''.

    \begin{Verbatim}[commandchars=\\\{\}]
{\color{incolor}In [{\color{incolor}3}]:} f\PYZob{}T<:Number\PYZcb{}(x::T,y::T) = 4x+10y
\end{Verbatim}

            \begin{Verbatim}[commandchars=\\\{\}]
{\color{outcolor}Out[{\color{outcolor}3}]:} f (generic function with 1 method)
\end{Verbatim}
        
    \begin{Verbatim}[commandchars=\\\{\}]
{\color{incolor}In [{\color{incolor}37}]:} f(2,3) \PYZsh{} 3x+2y since (::Int,::Int) is stricter
\end{Verbatim}

            \begin{Verbatim}[commandchars=\\\{\}]
{\color{outcolor}Out[{\color{outcolor}37}]:} 12
\end{Verbatim}
        
    \begin{Verbatim}[commandchars=\\\{\}]
{\color{incolor}In [{\color{incolor}4}]:} f(2.0,3.0) \PYZsh{} 4x+10y
\end{Verbatim}

            \begin{Verbatim}[commandchars=\\\{\}]
{\color{outcolor}Out[{\color{outcolor}4}]:} 38.0
\end{Verbatim}
        
    Note that type parameterizations can have as many types as possible, and
do not need to declare a supertype. For example, we can say that there
is an \texttt{x} which must be a Number, while \texttt{y} and \texttt{z}
must match types:

    \begin{Verbatim}[commandchars=\\\{\}]
{\color{incolor}In [{\color{incolor}5}]:} f\PYZob{}T<:Number,T2\PYZcb{}(x::T,y::T2,z::T2) = 5x + 5y + 5z
\end{Verbatim}

            \begin{Verbatim}[commandchars=\\\{\}]
{\color{outcolor}Out[{\color{outcolor}5}]:} f (generic function with 2 methods)
\end{Verbatim}
        
    We will go into more depth on multiple dispatch later since this is the
core design feature of Julia. The key feature is that Julia functions
specialize on the types of their arguments. This means that \texttt{f}
is a separately compiled function for each method (and for parametric
types, each possible method). The first time it is called it will
compile.

\begin{center}\rule{3in}{0.4pt}\end{center}

\paragraph{Question 2}\label{question-2}

Can you explain these timings?

    \begin{Verbatim}[commandchars=\\\{\}]
{\color{incolor}In [{\color{incolor}13}]:} f(x,y,z,w) = x+y+z+w
         @time f(1,1,1,1)
         @time f(1,1,1,1)
         @time f(1,1,1,1)
         @time f(1,1,1,1.0)
         @time f(1,1,1,1.0)
\end{Verbatim}

    \begin{Verbatim}[commandchars=\\\{\}]
  0.003228 seconds (257 allocations: 13.354 KB)
  0.000001 seconds (3 allocations: 144 bytes)
  0.000001 seconds (3 allocations: 144 bytes)
  0.002551 seconds (274 allocations: 14.007 KB)
  0.000001 seconds (4 allocations: 160 bytes)

    \end{Verbatim}

    \begin{Verbatim}[commandchars=\\\{\}]
WARNING: Method definition f(Any, Any, Any, Any) in module Main at In[12]:1 overwritten at In[13]:1.

    \end{Verbatim}

            \begin{Verbatim}[commandchars=\\\{\}]
{\color{outcolor}Out[{\color{outcolor}13}]:} 4.0
\end{Verbatim}
        
    \begin{center}\rule{3in}{0.4pt}\end{center}

Note that functions can also feature optional arguments:

    \begin{Verbatim}[commandchars=\\\{\}]
{\color{incolor}In [{\color{incolor}42}]:} function test\PYZus{}function(x,y;z=0) \PYZsh{}z is an optional argument
           if z==0
             return x+y,x*y \PYZsh{}Return a tuple
           else
           return x*y*z,x+y+z \PYZsh{}Return a different tuple
           \PYZsh{}whitespace is optional
           end \PYZsh{}End if statement
         end \PYZsh{}End function definition
\end{Verbatim}

            \begin{Verbatim}[commandchars=\\\{\}]
{\color{outcolor}Out[{\color{outcolor}42}]:} test\_function (generic function with 1 method)
\end{Verbatim}
        
    Here, if z is not specified, then it's 0.

    \begin{Verbatim}[commandchars=\\\{\}]
{\color{incolor}In [{\color{incolor}45}]:} x,y = test\PYZus{}function(1,2)
\end{Verbatim}

            \begin{Verbatim}[commandchars=\\\{\}]
{\color{outcolor}Out[{\color{outcolor}45}]:} (3,2)
\end{Verbatim}
        
    \begin{Verbatim}[commandchars=\\\{\}]
{\color{incolor}In [{\color{incolor}46}]:} x,y = test\PYZus{}function(1,2;z=3)
\end{Verbatim}

            \begin{Verbatim}[commandchars=\\\{\}]
{\color{outcolor}Out[{\color{outcolor}46}]:} (6,6)
\end{Verbatim}
        
    Notice that we also featured multiple return values.

    \begin{Verbatim}[commandchars=\\\{\}]
{\color{incolor}In [{\color{incolor}47}]:} println(x); println(y)
\end{Verbatim}

    \begin{Verbatim}[commandchars=\\\{\}]
6
6

    \end{Verbatim}

    The return type for multiple return values is a Tuple. The syntax for a
tuple is \texttt{(x,y,z,...)} or inside of functions you can use the
shorthand \texttt{x,y,z,...} as shown.

Note that functions in Julia are ``first-class''. This means that
functions are just a type themselves. Therefore functions can make
functions, you can store functions as variables, pass them as variables,
etc. For example:

    \begin{Verbatim}[commandchars=\\\{\}]
{\color{incolor}In [{\color{incolor}7}]:} function function\PYZus{}playtime(x) \PYZsh{}z is an optional argument
            y = 2+x
            function test()
                2y \PYZsh{} y is defined in the previous scope, so it's available here
            end
            z = test() * test()
            return z,test
        end \PYZsh{}End function definition
        z,test = function\PYZus{}playtime(2)
\end{Verbatim}

    \begin{Verbatim}[commandchars=\\\{\}]
WARNING: Method definition function\_playtime(Any) in module Main at In[6]:2 overwritten at In[7]:2.

    \end{Verbatim}

            \begin{Verbatim}[commandchars=\\\{\}]
{\color{outcolor}Out[{\color{outcolor}7}]:} (64,test)
\end{Verbatim}
        
    \begin{Verbatim}[commandchars=\\\{\}]
{\color{incolor}In [{\color{incolor}14}]:} test()
\end{Verbatim}

            \begin{Verbatim}[commandchars=\\\{\}]
{\color{outcolor}Out[{\color{outcolor}14}]:} 8
\end{Verbatim}
        
    Notice that \texttt{test()} does not get passed in \texttt{y} but knows
what \texttt{y} is. This is due to the function scoping rules: an inner
function can know the variables defined in the same scope as the
function. This rule is recursive, leading us to the conclusion that the
top level scope is global. Yes, that means

    \begin{Verbatim}[commandchars=\\\{\}]
{\color{incolor}In [{\color{incolor}18}]:} a = 2
\end{Verbatim}

            \begin{Verbatim}[commandchars=\\\{\}]
{\color{outcolor}Out[{\color{outcolor}18}]:} 2
\end{Verbatim}
        
    defines a global variable. We will go into more detail on this.

Lastly we show the anonymous function syntax. This allows you to define
a function inline.

    \begin{Verbatim}[commandchars=\\\{\}]
{\color{incolor}In [{\color{incolor}20}]:} g = (x,y) -> 2x+y
\end{Verbatim}

            \begin{Verbatim}[commandchars=\\\{\}]
{\color{outcolor}Out[{\color{outcolor}20}]:} (::\#5) (generic function with 1 method)
\end{Verbatim}
        
    Unlike named functions, \texttt{g} is simply a function in a variable
and can be overwritten at any time:

    \begin{Verbatim}[commandchars=\\\{\}]
{\color{incolor}In [{\color{incolor}21}]:} g = (x) -> 2x
\end{Verbatim}

            \begin{Verbatim}[commandchars=\\\{\}]
{\color{outcolor}Out[{\color{outcolor}21}]:} (::\#7) (generic function with 1 method)
\end{Verbatim}
        
    An anonymous function cannot have more than 1 dispatch. However, as of
v0.5, they are compiled and thus do not have any performance
disadvantages from named functions.

    \subsection{Type Declaration Syntax}\label{type-declaration-syntax}

A type is what in many other languages is an ``object''. If that is a
foreign concept, thing of a type as a thing which has named components.
A type is the idea for what the thing is, while an instantiation of the
type is a specific one. For example, you can think of a car as having an
make and a model. So that means a Toyota RAV4 is an instantiation of the
car type.

In Julia, we would define the car type as follows:

    \begin{Verbatim}[commandchars=\\\{\}]
{\color{incolor}In [{\color{incolor}48}]:} type Car
             make
             model
         end
\end{Verbatim}

    We could then make the instance of a car as follows:

    \begin{Verbatim}[commandchars=\\\{\}]
{\color{incolor}In [{\color{incolor}49}]:} mycar = Car("Toyota","Rav4")
\end{Verbatim}

            \begin{Verbatim}[commandchars=\\\{\}]
{\color{outcolor}Out[{\color{outcolor}49}]:} Car("Toyota","Rav4")
\end{Verbatim}
        
    Here I introduced the string syntax for Julia which uses ``\ldots{}''
(like most other languages, I'm glaring at you MATLAB). I can grab the
``fields'' of my type using the \texttt{.} syntax:

    \begin{Verbatim}[commandchars=\\\{\}]
{\color{incolor}In [{\color{incolor}51}]:} mycar.make
\end{Verbatim}

            \begin{Verbatim}[commandchars=\\\{\}]
{\color{outcolor}Out[{\color{outcolor}51}]:} "Toyota"
\end{Verbatim}
        
    To ``enhance Julia's performance'', one usually likes to make the typing
stricter. For example, we can define a WorkshopParticipant (notice the
convention for types is capital letters, CamelCase) as having a name and
a field. The name will be a string and the field will be a Symbol type,
(defined by :Symbol, which we will go into plenty more detail later).

    \begin{Verbatim}[commandchars=\\\{\}]
{\color{incolor}In [{\color{incolor}52}]:} type WorkshopParticipant
             name::String
             field::Symbol
         end
         tony = WorkshopParticipant("Tony",:physics)
\end{Verbatim}

            \begin{Verbatim}[commandchars=\\\{\}]
{\color{outcolor}Out[{\color{outcolor}52}]:} WorkshopParticipant("Tony",:physics)
\end{Verbatim}
        
    As with functions, types can be set ``parametrically''. For example, we
can have an StaffMember have a name and a field, but also an age. We can
allow this age to be any Number type as follows:

    \begin{Verbatim}[commandchars=\\\{\}]
{\color{incolor}In [{\color{incolor}1}]:} type StaffMember\PYZob{}T<:Number\PYZcb{}
            name::String
            field::Symbol
            age::T
        end
        ter = StaffMember("Terry",:football,17)
\end{Verbatim}

            \begin{Verbatim}[commandchars=\\\{\}]
{\color{outcolor}Out[{\color{outcolor}1}]:} StaffMember\{Int64\}("Terry",:football,17)
\end{Verbatim}
        
    The rules for parametric typing is the same as for functions. Note that
most of Julia's types, like Float64 and Int, are natively defined in
Julia in this manner. This means that there's no limit for user defined
types, only your imagination. Indeed, many of Julia's features first
start out as a prototyping package before it's ever moved into Base (the
Julia library that ships as the Base module in every installation).

Lastly, there exist abstract types. These types cannot be instantiated
but are used to build the type hierarchy. You've already seen one
abstract type, Number. We can define one for Person using the Abstract
keyword

    \begin{Verbatim}[commandchars=\\\{\}]
{\color{incolor}In [{\color{incolor}15}]:} abstract Person
\end{Verbatim}

    Then we can set types as a subtype of person

    \begin{Verbatim}[commandchars=\\\{\}]
{\color{incolor}In [{\color{incolor}16}]:} type Student <: Person
             grade
         end
\end{Verbatim}

    \begin{center}\rule{3in}{0.4pt}\end{center}

\paragraph{Question 3}\label{question-3}

Can you make a function \texttt{k(x)} where \texttt{x} has to be a
Person?

\begin{center}\rule{3in}{0.4pt}\end{center}

Another ``version'' of type is \texttt{immutable}. When one uses
\texttt{immutable}, the fields of the type cannot be changed. However,
Julia will automatically stack allocate immutable types, whereas
standard types are heap allocated. If this is unfamiliar terminology,
then think of this as meaning that immutable types are able to be stored
closer to the CPU and have less cost for memory access (this is a detail
not present in many scripting languages). Many things like Julia's
built-in Number types are defined as \texttt{immutable} in order to give
good performance.

    \begin{Verbatim}[commandchars=\\\{\}]
{\color{incolor}In [{\color{incolor}63}]:} immutable Field
             name
             school
         end
         ds = Field(:DataScience,[:PhysicalScience;:ComputerScience])
\end{Verbatim}

    \begin{Verbatim}[commandchars=\\\{\}]
WARNING: Method definition (::Type\{Main.Field\})(Any, Any) in module Main at In[58]:2 overwritten at In[63]:2.

    \end{Verbatim}

            \begin{Verbatim}[commandchars=\\\{\}]
{\color{outcolor}Out[{\color{outcolor}63}]:} Field(:DataScience,Symbol[:PhysicalScience,:ComputerScience])
\end{Verbatim}
        
    \begin{center}\rule{3in}{0.4pt}\end{center}

\paragraph{Question 4}\label{question-4}

Can you explain this interesting quirk? Thus Field is immutable, meaning
that \texttt{ds.name} and \texttt{ds.school} cannot be changed:

    \begin{Verbatim}[commandchars=\\\{\}]
{\color{incolor}In [{\color{incolor}64}]:} ds.name = :ComputationalStatistics
\end{Verbatim}

    \begin{Verbatim}[commandchars=\\\{\}]

        LoadError: type Field is immutable
    while loading In[64], in expression starting on line 1

        

    \end{Verbatim}

    However, the following is allowed:

    \begin{Verbatim}[commandchars=\\\{\}]
{\color{incolor}In [{\color{incolor}65}]:} push!(ds.school,:BiologicalScience)
         ds.school
\end{Verbatim}

            \begin{Verbatim}[commandchars=\\\{\}]
{\color{outcolor}Out[{\color{outcolor}65}]:} 3-element Array\{Symbol,1\}:
          :PhysicalScience  
          :ComputerScience  
          :BiologicalScience
\end{Verbatim}
        
    (Hint: recall that an array is not the values itself, but a pointer to
the memory of the values)

\begin{center}\rule{3in}{0.4pt}\end{center}

    One important detail in Julia is that everything is a type (and every
piece of code is an Expression type, more on this later). Thus functions
are also types, which we can access the fields of. Not only is
everything compiled down to C, but all of the ``C parts'' are always
accessible. For example, we can, if we so choose, get a function
pointer:

    \begin{Verbatim}[commandchars=\\\{\}]
{\color{incolor}In [{\color{incolor}46}]:} foo(x) = 2x
         first(methods(foo)).lambda\PYZus{}template.fptr
\end{Verbatim}

            \begin{Verbatim}[commandchars=\\\{\}]
{\color{outcolor}Out[{\color{outcolor}46}]:} Ptr\{Void\} @0x0000000000000000
\end{Verbatim}
        
    \subsection{Documentation and
``Hunting''}\label{documentation-and-hunting}

The main source of information is the
\href{http://docs.julialang.org/en/latest/manual/}{Julia Documentation}.
Julia also provides lots of built-in documentation and ways to find out
what's going on. The number of tools for ``hunting down what's going on
/ available'' is too numerous to explain in full detail here, so instead
this will just touch on what's important. For example, the ? gets you to
the documentation for a type, function, etc.

    \begin{Verbatim}[commandchars=\\\{\}]
{\color{incolor}In [{\color{incolor}23}]:} ?copy
\end{Verbatim}

    \begin{Verbatim}[commandchars=\\\{\}]
search: copy copy! copysign deepcopy unsafe\_copy! cospi complex Complex


    \end{Verbatim}
\texttt{\color{outcolor}Out[{\color{outcolor}23}]:}
    
    \begin{verbatim}
copy(x)
\end{verbatim}

Create a shallow copy of \texttt{x}: the outer structure is copied, but
not all internal values. For example, copying an array produces a new
array with identically-same elements as the original.

    

    To find out what methods are available, we can use the \texttt{methods}
function. For example, let's see how \texttt{+} is defined:

    \begin{Verbatim}[commandchars=\\\{\}]
{\color{incolor}In [{\color{incolor}25}]:} methods(+)
\end{Verbatim}

            \begin{Verbatim}[commandchars=\\\{\}]
{\color{outcolor}Out[{\color{outcolor}25}]:} \# 163 methods for generic function "+":
         +(x::Bool, z::Complex\{Bool\}) at complex.jl:136
         +(x::Bool, y::Bool) at bool.jl:48
         +(x::Bool) at bool.jl:45
         +\{T<:AbstractFloat\}(x::Bool, y::T) at bool.jl:55
         +(x::Bool, z::Complex) at complex.jl:143
         +(x::Bool, A::AbstractArray\{Bool,N<:Any\}) at arraymath.jl:91
         +(x::Float32, y::Float32) at float.jl:239
         +(x::Float64, y::Float64) at float.jl:240
         +(z::Complex\{Bool\}, x::Bool) at complex.jl:137
         +(z::Complex\{Bool\}, x::Real) at complex.jl:151
         +(a::Float16, b::Float16) at float16.jl:136
         +(x::Char, y::Integer) at char.jl:40
         +(c::BigInt, x::BigFloat) at mpfr.jl:240
         +(a::BigInt, b::BigInt, c::BigInt, d::BigInt, e::BigInt) at gmp.jl:298
         +(a::BigInt, b::BigInt, c::BigInt, d::BigInt) at gmp.jl:291
         +(a::BigInt, b::BigInt, c::BigInt) at gmp.jl:285
         +(x::BigInt, y::BigInt) at gmp.jl:255
         +(x::BigInt, c::Union\{UInt16,UInt32,UInt64,UInt8\}) at gmp.jl:310
         +(x::BigInt, c::Union\{Int16,Int32,Int64,Int8\}) at gmp.jl:326
         +(a::BigFloat, b::BigFloat, c::BigFloat, d::BigFloat, e::BigFloat) at mpfr.jl:388
         +(a::BigFloat, b::BigFloat, c::BigFloat, d::BigFloat) at mpfr.jl:381
         +(a::BigFloat, b::BigFloat, c::BigFloat) at mpfr.jl:375
         +(x::BigFloat, c::BigInt) at mpfr.jl:236
         +(x::BigFloat, y::BigFloat) at mpfr.jl:205
         +(x::BigFloat, c::Union\{UInt16,UInt32,UInt64,UInt8\}) at mpfr.jl:212
         +(x::BigFloat, c::Union\{Int16,Int32,Int64,Int8\}) at mpfr.jl:220
         +(x::BigFloat, c::Union\{Float16,Float32,Float64\}) at mpfr.jl:228
         +\{T\}(B::BitArray\{2\}, J::UniformScaling\{T\}) at linalg/uniformscaling.jl:38
         +(a::Base.Pkg.Resolve.VersionWeights.VWPreBuildItem, b::Base.Pkg.Resolve.VersionWeights.VWPreBuildItem) at pkg/resolve/versionweight.jl:85
         +(a::Base.Pkg.Resolve.VersionWeights.VWPreBuild, b::Base.Pkg.Resolve.VersionWeights.VWPreBuild) at pkg/resolve/versionweight.jl:131
         +(a::Base.Pkg.Resolve.VersionWeights.VersionWeight, b::Base.Pkg.Resolve.VersionWeights.VersionWeight) at pkg/resolve/versionweight.jl:185
         +(a::Base.Pkg.Resolve.MaxSum.FieldValues.FieldValue, b::Base.Pkg.Resolve.MaxSum.FieldValues.FieldValue) at pkg/resolve/fieldvalue.jl:44
         +(x::Base.Dates.CompoundPeriod, y::Base.Dates.CompoundPeriod) at dates/periods.jl:314
         +(x::Base.Dates.CompoundPeriod, y::Base.Dates.Period) at dates/periods.jl:312
         +(x::Base.Dates.CompoundPeriod, y::Base.Dates.TimeType) at dates/periods.jl:359
         +(dt::DateTime, z::Base.Dates.Month) at dates/arithmetic.jl:37
         +(dt::DateTime, y::Base.Dates.Year) at dates/arithmetic.jl:13
         +(x::DateTime, y::Base.Dates.Period) at dates/arithmetic.jl:64
         +(x::Date, y::Base.Dates.Day) at dates/arithmetic.jl:62
         +(x::Date, y::Base.Dates.Week) at dates/arithmetic.jl:60
         +(dt::Date, z::Base.Dates.Month) at dates/arithmetic.jl:43
         +(dt::Date, y::Base.Dates.Year) at dates/arithmetic.jl:17
         +(y::AbstractFloat, x::Bool) at bool.jl:57
         +\{T<:Union\{Int128,Int16,Int32,Int64,Int8,UInt128,UInt16,UInt32,UInt64,UInt8\}\}(x::T, y::T) at int.jl:32
         +(x::Integer, y::Ptr) at pointer.jl:108
         +(z::Complex, w::Complex) at complex.jl:125
         +(z::Complex, x::Bool) at complex.jl:144
         +(x::Real, z::Complex\{Bool\}) at complex.jl:150
         +(x::Real, z::Complex) at complex.jl:162
         +(z::Complex, x::Real) at complex.jl:163
         +(x::Rational, y::Rational) at rational.jl:199
         +(x::Integer, y::Char) at char.jl:41
         +\{N\}(i::Integer, index::CartesianIndex\{N\}) at multidimensional.jl:58
         +(c::Union\{UInt16,UInt32,UInt64,UInt8\}, x::BigInt) at gmp.jl:314
         +(c::Union\{Int16,Int32,Int64,Int8\}, x::BigInt) at gmp.jl:327
         +(c::Union\{UInt16,UInt32,UInt64,UInt8\}, x::BigFloat) at mpfr.jl:216
         +(c::Union\{Int16,Int32,Int64,Int8\}, x::BigFloat) at mpfr.jl:224
         +(c::Union\{Float16,Float32,Float64\}, x::BigFloat) at mpfr.jl:232
         +(x::Irrational, y::Irrational) at irrationals.jl:88
         +(x::Number) at operators.jl:115
         +\{T<:Number\}(x::T, y::T) at promotion.jl:255
         +(x::Number, y::Number) at promotion.jl:190
         +(r1::OrdinalRange, r2::OrdinalRange) at operators.jl:505
         +\{T<:AbstractFloat\}(r1::FloatRange\{T\}, r2::FloatRange\{T\}) at operators.jl:512
         +\{T<:AbstractFloat\}(r1::LinSpace\{T\}, r2::LinSpace\{T\}) at operators.jl:531
         +(r1::Union\{FloatRange,LinSpace,OrdinalRange\}, r2::Union\{FloatRange,LinSpace,OrdinalRange\}) at operators.jl:544
         +(x::Ptr, y::Integer) at pointer.jl:106
         +(A::BitArray, B::BitArray) at bitarray.jl:1042
         +(A::Array\{T<:Any,2\}, B::Diagonal) at linalg/special.jl:121
         +(A::Array\{T<:Any,2\}, B::Bidiagonal) at linalg/special.jl:121
         +(A::Array\{T<:Any,2\}, B::Tridiagonal) at linalg/special.jl:121
         +(A::Array\{T<:Any,2\}, B::SymTridiagonal) at linalg/special.jl:130
         +(A::Array\{T<:Any,2\}, B::Base.LinAlg.AbstractTriangular) at linalg/special.jl:158
         +(A::Array, B::SparseMatrixCSC) at sparse/sparsematrix.jl:1711
         +\{P<:Union\{Base.Dates.CompoundPeriod,Base.Dates.Period\}\}(x::Union\{Base.ReshapedArray\{P,N<:Any,A<:DenseArray,MI<:Tuple\{Vararg\{Base.MultiplicativeInverses.SignedMultiplicativeInverse\{Int64\},N<:Any\}\}\},DenseArray\{P,N<:Any\},SubArray\{P,N<:Any,A<:Union\{Base.ReshapedArray\{T<:Any,N<:Any,A<:DenseArray,MI<:Tuple\{Vararg\{Base.MultiplicativeInverses.SignedMultiplicativeInverse\{Int64\},N<:Any\}\}\},DenseArray\},I<:Tuple\{Vararg\{Union\{Base.AbstractCartesianIndex,Colon,Int64,Range\{Int64\}\},N<:Any\}\},L<:Any\}\}) at dates/periods.jl:323
         +(A::AbstractArray\{Bool,N<:Any\}, x::Bool) at arraymath.jl:90
         +(A::SymTridiagonal, B::SymTridiagonal) at linalg/tridiag.jl:96
         +(A::Tridiagonal, B::Tridiagonal) at linalg/tridiag.jl:494
         +(A::UpperTriangular, B::UpperTriangular) at linalg/triangular.jl:357
         +(A::LowerTriangular, B::LowerTriangular) at linalg/triangular.jl:358
         +(A::UpperTriangular, B::Base.LinAlg.UnitUpperTriangular) at linalg/triangular.jl:359
         +(A::LowerTriangular, B::Base.LinAlg.UnitLowerTriangular) at linalg/triangular.jl:360
         +(A::Base.LinAlg.UnitUpperTriangular, B::UpperTriangular) at linalg/triangular.jl:361
         +(A::Base.LinAlg.UnitLowerTriangular, B::LowerTriangular) at linalg/triangular.jl:362
         +(A::Base.LinAlg.UnitUpperTriangular, B::Base.LinAlg.UnitUpperTriangular) at linalg/triangular.jl:363
         +(A::Base.LinAlg.UnitLowerTriangular, B::Base.LinAlg.UnitLowerTriangular) at linalg/triangular.jl:364
         +(A::Base.LinAlg.AbstractTriangular, B::Base.LinAlg.AbstractTriangular) at linalg/triangular.jl:365
         +(Da::Diagonal, Db::Diagonal) at linalg/diagonal.jl:110
         +(A::Bidiagonal, B::Bidiagonal) at linalg/bidiag.jl:256
         +(UL::UpperTriangular, J::UniformScaling) at linalg/uniformscaling.jl:55
         +(UL::Base.LinAlg.UnitUpperTriangular, J::UniformScaling) at linalg/uniformscaling.jl:58
         +(UL::LowerTriangular, J::UniformScaling) at linalg/uniformscaling.jl:55
         +(UL::Base.LinAlg.UnitLowerTriangular, J::UniformScaling) at linalg/uniformscaling.jl:58
         +(A::Diagonal, B::Bidiagonal) at linalg/special.jl:120
         +(A::Bidiagonal, B::Diagonal) at linalg/special.jl:121
         +(A::Diagonal, B::Tridiagonal) at linalg/special.jl:120
         +(A::Tridiagonal, B::Diagonal) at linalg/special.jl:121
         +(A::Diagonal, B::Array\{T<:Any,2\}) at linalg/special.jl:120
         +(A::Bidiagonal, B::Tridiagonal) at linalg/special.jl:120
         +(A::Tridiagonal, B::Bidiagonal) at linalg/special.jl:121
         +(A::Bidiagonal, B::Array\{T<:Any,2\}) at linalg/special.jl:120
         +(A::Tridiagonal, B::Array\{T<:Any,2\}) at linalg/special.jl:120
         +(A::SymTridiagonal, B::Tridiagonal) at linalg/special.jl:129
         +(A::Tridiagonal, B::SymTridiagonal) at linalg/special.jl:130
         +(A::SymTridiagonal, B::Array\{T<:Any,2\}) at linalg/special.jl:129
         +(A::Diagonal, B::SymTridiagonal) at linalg/special.jl:138
         +(A::SymTridiagonal, B::Diagonal) at linalg/special.jl:139
         +(A::Bidiagonal, B::SymTridiagonal) at linalg/special.jl:138
         +(A::SymTridiagonal, B::Bidiagonal) at linalg/special.jl:139
         +(A::Diagonal, B::UpperTriangular) at linalg/special.jl:150
         +(A::UpperTriangular, B::Diagonal) at linalg/special.jl:151
         +(A::Diagonal, B::Base.LinAlg.UnitUpperTriangular) at linalg/special.jl:150
         +(A::Base.LinAlg.UnitUpperTriangular, B::Diagonal) at linalg/special.jl:151
         +(A::Diagonal, B::LowerTriangular) at linalg/special.jl:150
         +(A::LowerTriangular, B::Diagonal) at linalg/special.jl:151
         +(A::Diagonal, B::Base.LinAlg.UnitLowerTriangular) at linalg/special.jl:150
         +(A::Base.LinAlg.UnitLowerTriangular, B::Diagonal) at linalg/special.jl:151
         +(A::Base.LinAlg.AbstractTriangular, B::SymTridiagonal) at linalg/special.jl:157
         +(A::SymTridiagonal, B::Base.LinAlg.AbstractTriangular) at linalg/special.jl:158
         +(A::Base.LinAlg.AbstractTriangular, B::Tridiagonal) at linalg/special.jl:157
         +(A::Tridiagonal, B::Base.LinAlg.AbstractTriangular) at linalg/special.jl:158
         +(A::Base.LinAlg.AbstractTriangular, B::Bidiagonal) at linalg/special.jl:157
         +(A::Bidiagonal, B::Base.LinAlg.AbstractTriangular) at linalg/special.jl:158
         +(A::Base.LinAlg.AbstractTriangular, B::Array\{T<:Any,2\}) at linalg/special.jl:157
         +\{P<:Union\{Base.Dates.CompoundPeriod,Base.Dates.Period\}\}(Y::Union\{Base.ReshapedArray\{P,N<:Any,A<:DenseArray,MI<:Tuple\{Vararg\{Base.MultiplicativeInverses.SignedMultiplicativeInverse\{Int64\},N<:Any\}\}\},DenseArray\{P,N<:Any\},SubArray\{P,N<:Any,A<:Union\{Base.ReshapedArray\{T<:Any,N<:Any,A<:DenseArray,MI<:Tuple\{Vararg\{Base.MultiplicativeInverses.SignedMultiplicativeInverse\{Int64\},N<:Any\}\}\},DenseArray\},I<:Tuple\{Vararg\{Union\{Base.AbstractCartesianIndex,Colon,Int64,Range\{Int64\}\},N<:Any\}\},L<:Any\}\}, x::Union\{Base.Dates.CompoundPeriod,Base.Dates.Period\}) at dates/periods.jl:337
         +\{P<:Union\{Base.Dates.CompoundPeriod,Base.Dates.Period\},Q<:Union\{Base.Dates.CompoundPeriod,Base.Dates.Period\}\}(X::Union\{Base.ReshapedArray\{P,N<:Any,A<:DenseArray,MI<:Tuple\{Vararg\{Base.MultiplicativeInverses.SignedMultiplicativeInverse\{Int64\},N<:Any\}\}\},DenseArray\{P,N<:Any\},SubArray\{P,N<:Any,A<:Union\{Base.ReshapedArray\{T<:Any,N<:Any,A<:DenseArray,MI<:Tuple\{Vararg\{Base.MultiplicativeInverses.SignedMultiplicativeInverse\{Int64\},N<:Any\}\}\},DenseArray\},I<:Tuple\{Vararg\{Union\{Base.AbstractCartesianIndex,Colon,Int64,Range\{Int64\}\},N<:Any\}\},L<:Any\}\}, Y::Union\{Base.ReshapedArray\{Q,N<:Any,A<:DenseArray,MI<:Tuple\{Vararg\{Base.MultiplicativeInverses.SignedMultiplicativeInverse\{Int64\},N<:Any\}\}\},DenseArray\{Q,N<:Any\},SubArray\{Q,N<:Any,A<:Union\{Base.ReshapedArray\{T<:Any,N<:Any,A<:DenseArray,MI<:Tuple\{Vararg\{Base.MultiplicativeInverses.SignedMultiplicativeInverse\{Int64\},N<:Any\}\}\},DenseArray\},I<:Tuple\{Vararg\{Union\{Base.AbstractCartesianIndex,Colon,Int64,Range\{Int64\}\},N<:Any\}\},L<:Any\}\}) at dates/periods.jl:338
         +\{T<:Base.Dates.TimeType,P<:Union\{Base.Dates.CompoundPeriod,Base.Dates.Period\}\}(x::Union\{Base.ReshapedArray\{P,N<:Any,A<:DenseArray,MI<:Tuple\{Vararg\{Base.MultiplicativeInverses.SignedMultiplicativeInverse\{Int64\},N<:Any\}\}\},DenseArray\{P,N<:Any\},SubArray\{P,N<:Any,A<:Union\{Base.ReshapedArray\{T<:Any,N<:Any,A<:DenseArray,MI<:Tuple\{Vararg\{Base.MultiplicativeInverses.SignedMultiplicativeInverse\{Int64\},N<:Any\}\}\},DenseArray\},I<:Tuple\{Vararg\{Union\{Base.AbstractCartesianIndex,Colon,Int64,Range\{Int64\}\},N<:Any\}\},L<:Any\}\}, y::T) at dates/arithmetic.jl:83
         +\{T<:Base.Dates.TimeType\}(r::Range\{T\}, x::Base.Dates.Period) at dates/ranges.jl:39
         +\{Tv1,Ti1,Tv2,Ti2\}(A\_1::SparseMatrixCSC\{Tv1,Ti1\}, A\_2::SparseMatrixCSC\{Tv2,Ti2\}) at sparse/sparsematrix.jl:1697
         +(A::SparseMatrixCSC, B::Array) at sparse/sparsematrix.jl:1709
         +(A::SparseMatrixCSC, J::UniformScaling) at sparse/sparsematrix.jl:3811
         +(x::AbstractSparseArray\{Tv<:Any,Ti<:Any,1\}, y::AbstractSparseArray\{Tv<:Any,Ti<:Any,1\}) at sparse/sparsevector.jl:1179
         +(x::Union\{Base.ReshapedArray\{T<:Any,1,A<:DenseArray,MI<:Tuple\{Vararg\{Base.MultiplicativeInverses.SignedMultiplicativeInverse\{Int64\},N<:Any\}\}\},DenseArray\{T<:Any,1\},SubArray\{T<:Any,1,A<:Union\{Base.ReshapedArray\{T<:Any,N<:Any,A<:DenseArray,MI<:Tuple\{Vararg\{Base.MultiplicativeInverses.SignedMultiplicativeInverse\{Int64\},N<:Any\}\}\},DenseArray\},I<:Tuple\{Vararg\{Union\{Base.AbstractCartesianIndex,Colon,Int64,Range\{Int64\}\},N<:Any\}\},L<:Any\}\}, y::AbstractSparseArray\{Tv<:Any,Ti<:Any,1\}) at sparse/sparsevector.jl:1180
         +(x::AbstractSparseArray\{Tv<:Any,Ti<:Any,1\}, y::Union\{Base.ReshapedArray\{T<:Any,1,A<:DenseArray,MI<:Tuple\{Vararg\{Base.MultiplicativeInverses.SignedMultiplicativeInverse\{Int64\},N<:Any\}\}\},DenseArray\{T<:Any,1\},SubArray\{T<:Any,1,A<:Union\{Base.ReshapedArray\{T<:Any,N<:Any,A<:DenseArray,MI<:Tuple\{Vararg\{Base.MultiplicativeInverses.SignedMultiplicativeInverse\{Int64\},N<:Any\}\}\},DenseArray\},I<:Tuple\{Vararg\{Union\{Base.AbstractCartesianIndex,Colon,Int64,Range\{Int64\}\},N<:Any\}\},L<:Any\}\}) at sparse/sparsevector.jl:1181
         +\{T<:Number\}(x::AbstractArray\{T,N<:Any\}) at abstractarraymath.jl:91
         +\{R,S\}(A::AbstractArray\{R,N<:Any\}, B::AbstractArray\{S,N<:Any\}) at arraymath.jl:49
         +(A::AbstractArray, x::Number) at arraymath.jl:94
         +(x::Number, A::AbstractArray) at arraymath.jl:95
         +\{N\}(index1::CartesianIndex\{N\}, index2::CartesianIndex\{N\}) at multidimensional.jl:52
         +\{N\}(index::CartesianIndex\{N\}, i::Integer) at multidimensional.jl:57
         +(J1::UniformScaling, J2::UniformScaling) at linalg/uniformscaling.jl:37
         +(J::UniformScaling, B::BitArray\{2\}) at linalg/uniformscaling.jl:39
         +(J::UniformScaling, A::AbstractArray\{T<:Any,2\}) at linalg/uniformscaling.jl:40
         +(J::UniformScaling, x::Number) at linalg/uniformscaling.jl:41
         +(x::Number, J::UniformScaling) at linalg/uniformscaling.jl:42
         +\{TA,TJ\}(A::AbstractArray\{TA,2\}, J::UniformScaling\{TJ\}) at linalg/uniformscaling.jl:102
         +\{T\}(a::Base.Pkg.Resolve.VersionWeights.HierarchicalValue\{T\}, b::Base.Pkg.Resolve.VersionWeights.HierarchicalValue\{T\}) at pkg/resolve/versionweight.jl:23
         +\{P<:Base.Dates.Period\}(x::P, y::P) at dates/periods.jl:70
         +(x::Base.Dates.Period, y::Base.Dates.Period) at dates/periods.jl:311
         +(y::Base.Dates.Period, x::Base.Dates.CompoundPeriod) at dates/periods.jl:313
         +(y::Base.Dates.Period, x::Base.Dates.TimeType) at dates/arithmetic.jl:66
         +\{T<:Base.Dates.TimeType\}(x::Base.Dates.Period, r::Range\{T\}) at dates/ranges.jl:40
         +(x::Union\{Base.Dates.CompoundPeriod,Base.Dates.Period\}) at dates/periods.jl:322
         +\{P<:Union\{Base.Dates.CompoundPeriod,Base.Dates.Period\}\}(x::Union\{Base.Dates.CompoundPeriod,Base.Dates.Period\}, Y::Union\{Base.ReshapedArray\{P,N<:Any,A<:DenseArray,MI<:Tuple\{Vararg\{Base.MultiplicativeInverses.SignedMultiplicativeInverse\{Int64\},N<:Any\}\}\},DenseArray\{P,N<:Any\},SubArray\{P,N<:Any,A<:Union\{Base.ReshapedArray\{T<:Any,N<:Any,A<:DenseArray,MI<:Tuple\{Vararg\{Base.MultiplicativeInverses.SignedMultiplicativeInverse\{Int64\},N<:Any\}\}\},DenseArray\},I<:Tuple\{Vararg\{Union\{Base.AbstractCartesianIndex,Colon,Int64,Range\{Int64\}\},N<:Any\}\},L<:Any\}\}) at dates/periods.jl:336
         +(x::Base.Dates.TimeType) at dates/arithmetic.jl:8
         +(a::Base.Dates.TimeType, b::Base.Dates.Period, c::Base.Dates.Period) at dates/periods.jl:348
         +(a::Base.Dates.TimeType, b::Base.Dates.Period, c::Base.Dates.Period, d::Base.Dates.Period{\ldots}) at dates/periods.jl:350
         +(x::Base.Dates.TimeType, y::Base.Dates.CompoundPeriod) at dates/periods.jl:354
         +(x::Base.Dates.Instant) at dates/arithmetic.jl:4
         +\{T<:Base.Dates.TimeType\}(x::AbstractArray\{T,N<:Any\}, y::Union\{Base.Dates.CompoundPeriod,Base.Dates.Period\}) at dates/arithmetic.jl:76
         +\{T<:Base.Dates.TimeType\}(y::Union\{Base.Dates.CompoundPeriod,Base.Dates.Period\}, x::AbstractArray\{T,N<:Any\}) at dates/arithmetic.jl:77
         +\{P<:Union\{Base.Dates.CompoundPeriod,Base.Dates.Period\}\}(y::Base.Dates.TimeType, x::Union\{Base.ReshapedArray\{P,N<:Any,A<:DenseArray,MI<:Tuple\{Vararg\{Base.MultiplicativeInverses.SignedMultiplicativeInverse\{Int64\},N<:Any\}\}\},DenseArray\{P,N<:Any\},SubArray\{P,N<:Any,A<:Union\{Base.ReshapedArray\{T<:Any,N<:Any,A<:DenseArray,MI<:Tuple\{Vararg\{Base.MultiplicativeInverses.SignedMultiplicativeInverse\{Int64\},N<:Any\}\}\},DenseArray\},I<:Tuple\{Vararg\{Union\{Base.AbstractCartesianIndex,Colon,Int64,Range\{Int64\}\},N<:Any\}\},L<:Any\}\}) at dates/arithmetic.jl:84
         +(a, b, c, xs{\ldots}) at operators.jl:138
\end{Verbatim}
        
    We can inspect a type by finding its fields with \texttt{fieldnames}

    \begin{Verbatim}[commandchars=\\\{\}]
{\color{incolor}In [{\color{incolor}40}]:} fieldnames(LinSpace)
\end{Verbatim}

            \begin{Verbatim}[commandchars=\\\{\}]
{\color{outcolor}Out[{\color{outcolor}40}]:} 4-element Array\{Symbol,1\}:
          :start  
          :stop   
          :len    
          :divisor
\end{Verbatim}
        
    and find out which method was used with the \texttt{@which} macro:

    \begin{Verbatim}[commandchars=\\\{\}]
{\color{incolor}In [{\color{incolor}43}]:} @which copy([1,2,3])
\end{Verbatim}

            \begin{Verbatim}[commandchars=\\\{\}]
{\color{outcolor}Out[{\color{outcolor}43}]:} copy\{T<:Array\{T,N\}\}(a::T) at array.jl:70
\end{Verbatim}
        
    Notice that this gives you a link to the source code where the function
is defined.

    Lastly, we can find out what type a variable is with the \texttt{typeof}
function:

    \begin{Verbatim}[commandchars=\\\{\}]
{\color{incolor}In [{\color{incolor}44}]:} a = [1;2;3]
         typeof(a)
\end{Verbatim}

            \begin{Verbatim}[commandchars=\\\{\}]
{\color{outcolor}Out[{\color{outcolor}44}]:} Array\{Int64,1\}
\end{Verbatim}
        
    \subsection{Some Basic Types}\label{some-basic-types}

Julia provides many basic types. Indeed, you will come to know Julia as
a system of multiple dispatch on types, meaning that the interaction of
types with functions is core to the design.

\subsubsection{Lazy Iterator Types}\label{lazy-iterator-types}

While MATLAB or Python has easy functions for building arrays, Julia
tends to side-step the actual ``array'' part with specially made types.
One such example are ranges. To define a range, use the
\texttt{start:stepsize:end} syntax. For example:

    \begin{Verbatim}[commandchars=\\\{\}]
{\color{incolor}In [{\color{incolor}45}]:} a = 1:5
         println(a)
         b = 1:2:10
         println(b)
\end{Verbatim}

    \begin{Verbatim}[commandchars=\\\{\}]
1:5
1:2:9

    \end{Verbatim}

    We can use them like any array. For example:

    \begin{Verbatim}[commandchars=\\\{\}]
{\color{incolor}In [{\color{incolor}47}]:} println(a[2]); println(b[3])
\end{Verbatim}

    \begin{Verbatim}[commandchars=\\\{\}]
2
5

    \end{Verbatim}

    But what is \texttt{b}?

    \begin{Verbatim}[commandchars=\\\{\}]
{\color{incolor}In [{\color{incolor}50}]:} println(typeof(b))
\end{Verbatim}

    \begin{Verbatim}[commandchars=\\\{\}]
StepRange\{Int64,Int64\}

    \end{Verbatim}

    \texttt{b} isn't an array, it's a StepRange. A StepRange has the ability
to act like an array using its fields:

    \begin{Verbatim}[commandchars=\\\{\}]
{\color{incolor}In [{\color{incolor}52}]:} fieldnames(StepRange)
\end{Verbatim}

            \begin{Verbatim}[commandchars=\\\{\}]
{\color{outcolor}Out[{\color{outcolor}52}]:} 3-element Array\{Symbol,1\}:
          :start
          :step 
          :stop 
\end{Verbatim}
        
    \paragraph{Question 4}\label{question-4}

If you know \texttt{start}, \texttt{step}, and \texttt{stop}, how do you
calculate the \texttt{i}th value? Can you create a function MyRange
which where for \texttt{a} being a \texttt{MyRange},
\texttt{value(a,i) == a{[}i{]}} for \texttt{a} a \texttt{StepRange}? (I
will show you how to make the \texttt{a{[}i{]}} syntax available for
your own types later. As noted before, all of these Julia types are
implemented in Julia, and therefore you have the power to implement it
yourself).

Note that at any time we can get the array from these kinds of type via
the \texttt{collect} function:

    \begin{Verbatim}[commandchars=\\\{\}]
{\color{incolor}In [{\color{incolor}55}]:} c = collect(a)
\end{Verbatim}

            \begin{Verbatim}[commandchars=\\\{\}]
{\color{outcolor}Out[{\color{outcolor}55}]:} 5-element Array\{Int64,1\}:
          1
          2
          3
          4
          5
\end{Verbatim}
        
    The reason why lazy iterator types are preferred is that they do not do
the computations until it's absolutely necessary, and they take up much
less space. We can check this with \texttt{@time}:

    \begin{Verbatim}[commandchars=\\\{\}]
{\color{incolor}In [{\color{incolor}73}]:} @time a = 1:100000
         @time a = 1:100
         @time b = collect(1:100000);
\end{Verbatim}

    \begin{Verbatim}[commandchars=\\\{\}]
  0.000003 seconds (5 allocations: 192 bytes)
  0.000001 seconds (5 allocations: 192 bytes)
  0.000076 seconds (8 allocations: 781.547 KB)

    \end{Verbatim}

    Notice that the amount of time the range takes is much shorter. This is
mostly because there is a lot less memory allocation needed: only a
\texttt{StepRange} is built, and all that holds is the three numbers.
However, \texttt{b} has to hold \texttt{100000} numbers, leading to the
huge difference.

\subsubsection{Dictionaries}\label{dictionaries}

Another common type is the Dictionary. It allows you to access
(key,value) pairs in a named manner. For example:

    \begin{Verbatim}[commandchars=\\\{\}]
{\color{incolor}In [{\color{incolor}1}]:} d = Dict(:test=>2,"silly"=>:suit)
        println(d[:test])
        println(d["silly"])
\end{Verbatim}

    \begin{Verbatim}[commandchars=\\\{\}]
2
suit

    \end{Verbatim}

    \subsubsection{Tuples}\label{tuples}

Tuples are immutable arrays. That means they can't be changed. However,
they are super fast. They are made with the \texttt{(x,y,z,...)} syntax
and are the standard return type of functions which return more than one
object.

    \begin{Verbatim}[commandchars=\\\{\}]
{\color{incolor}In [{\color{incolor}4}]:} tup = (2.,3) \PYZsh{} Don't have to match types
        x,y = (3.0,"hi") \PYZsh{} Can separate a tuple to multiple variables
\end{Verbatim}

            \begin{Verbatim}[commandchars=\\\{\}]
{\color{outcolor}Out[{\color{outcolor}4}]:} (3.0,"hi")
\end{Verbatim}
        
    \subsection{Package Management}\label{package-management}

Julia's package system is similar to R/Python in that a large number of
packages are freely available. You search for them in places like
\href{http://genieframework.com/packages}{Julia's Package Genie}, or
from the \href{http://pkg.julialang.org/}{Julia Package Listing}. Let's
take a look at the \href{https://github.com/tbreloff/Plots.jl}{Plots.jl
package by Tom Breloff}. This is a highly reguarded plotting package
which we will make extensive use of later in this workshop. To add a
package, use \texttt{Pkg.add}

    \begin{Verbatim}[commandchars=\\\{\}]
{\color{incolor}In [{\color{incolor}1}]:} Pkg.update() \PYZsh{} You may need to update your local packages first
        Pkg.add("Plots")
\end{Verbatim}

    \begin{Verbatim}[commandchars=\\\{\}]
INFO: Initializing package repository /home/juser/.julia/v0.5
INFO: Cloning METADATA from https://github.com/JuliaLang/METADATA.jl
INFO: Cloning cache of ColorTypes from https://github.com/JuliaGraphics/ColorTypes.jl.git
INFO: Cloning cache of Colors from https://github.com/JuliaGraphics/Colors.jl.git
INFO: Cloning cache of Compat from https://github.com/JuliaLang/Compat.jl.git
INFO: Cloning cache of FixedPointNumbers from https://github.com/JeffBezanson/FixedPointNumbers.jl.git
INFO: Cloning cache of FixedSizeArrays from https://github.com/SimonDanisch/FixedSizeArrays.jl.git
INFO: Cloning cache of Iterators from https://github.com/JuliaLang/Iterators.jl.git
INFO: Cloning cache of Measures from https://github.com/dcjones/Measures.jl.git
INFO: Cloning cache of PlotUtils from https://github.com/JuliaPlots/PlotUtils.jl.git
INFO: Cloning cache of Plots from https://github.com/tbreloff/Plots.jl.git
INFO: Cloning cache of RecipesBase from https://github.com/JuliaPlots/RecipesBase.jl.git
INFO: Cloning cache of Reexport from https://github.com/simonster/Reexport.jl.git
INFO: Cloning cache of Showoff from https://github.com/JuliaGraphics/Showoff.jl.git
INFO: Installing ColorTypes v0.2.6
INFO: Installing Colors v0.6.7
INFO: Installing Compat v0.9.2
INFO: Installing FixedPointNumbers v0.1.6
INFO: Installing FixedSizeArrays v0.2.3
INFO: Installing Iterators v0.1.10
INFO: Installing Measures v0.0.3
INFO: Installing PlotUtils v0.0.4
INFO: Installing Plots v0.9.1
INFO: Installing RecipesBase v0.0.6
INFO: Installing Reexport v0.0.3
INFO: Installing Showoff v0.0.7
INFO: Building Plots
INFO: Cannot find deps/plotly-latest.min.js{\ldots} downloading latest version.
  \% Total    \% Received \% Xferd  Average Speed   Time    Time     Time  Current
                                 Dload  Upload   Total   Spent    Left  Speed
100 1746k  100 1746k    0     0  9420k      0 --:--:-- --:--:-- --:--:-- 9439k
INFO: Package database updated

    \end{Verbatim}

    This will install the package to your local system. However, this will
only work for registered packages. To add a non-registered package, go
to the Github repository to find the clone URL and use
\texttt{Pkg.clone}. For example, to install the
\texttt{ParameterizedFunctions} package, we can use:

    \begin{Verbatim}[commandchars=\\\{\}]
{\color{incolor}In [{\color{incolor}1}]:} Pkg.clone("https://github.com/JuliaDiffEq/ParameterizedFunctions.jl")
\end{Verbatim}

    \begin{Verbatim}[commandchars=\\\{\}]
INFO: Cloning ParameterizedFunctions from https://github.com/JuliaDiffEq/ParameterizedFunctions.jl
INFO: Computing changes{\ldots}
WARNING: julia is fixed at 0.5.0-rc4+0 conflicting with requirement for ParameterizedFunctions: [0.5.0,∞)
INFO: Cloning cache of BinDeps from https://github.com/JuliaLang/BinDeps.jl.git
INFO: Cloning cache of Conda from https://github.com/JuliaPy/Conda.jl.git
INFO: Cloning cache of JSON from https://github.com/JuliaLang/JSON.jl.git
INFO: Cloning cache of MacroTools from https://github.com/MikeInnes/MacroTools.jl.git
INFO: Cloning cache of PyCall from https://github.com/JuliaPy/PyCall.jl.git
INFO: Cloning cache of SHA from https://github.com/staticfloat/SHA.jl.git
INFO: Cloning cache of SymPy from https://github.com/JuliaPy/SymPy.jl.git
INFO: Cloning cache of URIParser from https://github.com/JuliaWeb/URIParser.jl.git
INFO: Installing BinDeps v0.4.5
INFO: Installing Conda v0.3.2
INFO: Installing JSON v0.7.0
INFO: Installing MacroTools v0.3.2
INFO: Installing PyCall v1.7.2
INFO: Installing SHA v0.2.1
INFO: Installing SymPy v0.3.1
INFO: Installing URIParser v0.1.6
INFO: Building PyCall
INFO: PyCall is using python (Python 2.7.6) at /usr/bin/python, libpython = libpython2.7

    \end{Verbatim}

    To use a package, you have to import the package. The \texttt{import}
statement will import the package without exporting the functions to the
namespace. For example:

    \begin{Verbatim}[commandchars=\\\{\}]
{\color{incolor}In [{\color{incolor}2}]:} import Plots
        Plots.plot(rand(4,4))
\end{Verbatim}

    \begin{Verbatim}[commandchars=\\\{\}]
INFO: Precompiling module Plots.
WARNING: Method definition cgrad(Any, Any) in module PlotUtils at /home/juser/.julia/v0.5/PlotUtils/src/color\_gradients.jl:82 overwritten at /home/juser/.julia/v0.5/PlotUtils/src/color\_gradients.jl:99.
WARNING: Method definition \#cgrad(Array\{Any, 1\}, PlotUtils.\#cgrad, Any, Any) in module PlotUtils overwritten.

    \end{Verbatim}

    
    
    \begin{Verbatim}[commandchars=\\\{\}]
[Plots.jl] Initializing backend: plotly

    \end{Verbatim}

    \begin{Verbatim}[commandchars=\\\{\}]
WARNING: Method definition show(IO, Base.Multimedia.MIME\{:text/plain\}, Plots.Plot) in module Plots at /home/juser/.julia/v0.5/Plots/src/output.jl:168 overwritten at /home/juser/.julia/v0.5/Plots/src/output.jl:241.

    \end{Verbatim}

    (Note that the first time a package is run, it will precompile a lot of
the functionality.) To instead export the functions (of the developers
choosing) to the namespace, we can use the \texttt{using} statement.
Since Plots.jl exports the \texttt{plot} command, we can then use it
without reference to the package that it came from:

    \begin{Verbatim}[commandchars=\\\{\}]
{\color{incolor}In [{\color{incolor}4}]:} using Plots
        plot(rand(4,4))
\end{Verbatim}

    What really makes this possible in Julia but not something like Python
is that namespace clashes are usually avoided by multiple dispatch. Most
packages will define their own types in order to use dispatches, and so
when they export the functionality, the methods are only for their own
types and thus do not clash with other packages. Therefore it's common
in Julia for concise syntax like \texttt{plot} to be part of packages,
all without fear of clashing.

Since Julia is currently under lots of development, you may wish to
checkout newer versions. By default, \texttt{Pkg.add} is the ``latest
release'', meaning the latest tagged version. However, the main version
shown in the Github repository is usually the ``master'' branch. It's
good development practice that the latest release is kept ``stable'',
while the ``master'' branch is kept ``working'', and development takes
place in another branch (many times labelled ``dev''). You can choose
which branch your local repository takes from. For example, to checkout
the master branch, we can use:

    \begin{Verbatim}[commandchars=\\\{\}]
{\color{incolor}In [{\color{incolor}6}]:} Pkg.checkout("Plots")
\end{Verbatim}

    \begin{Verbatim}[commandchars=\\\{\}]
INFO: Checking out Plots master{\ldots}
INFO: Pulling Plots latest master{\ldots}
INFO: No packages to install, update or remove

    \end{Verbatim}

    This will usually gives us pretty up to date features (if you are using
a ``unreleased version of Julia'' like building from the source of the
Julia nightly, you may need to checkout master in order to get some
packages working). However, to go to a specific branch we can give the
branch as another argument:

    \begin{Verbatim}[commandchars=\\\{\}]
{\color{incolor}In [{\color{incolor}8}]:} Pkg.checkout("Plots","dev")
\end{Verbatim}

    \begin{Verbatim}[commandchars=\\\{\}]
INFO: Checking out Plots dev{\ldots}
INFO: Pulling Plots latest dev{\ldots}
INFO: No packages to install, update or remove

    \end{Verbatim}

    This is not advised if you don't know what you're doing (i.e.~talk to
the developer or read the pull requests (PR)), but this is common if you
talk to a developer and they say ``yes, I already implemented that.
Checkout the dev branch and use \texttt{plot(...)}.

    \subsection{Metaprogramming}\label{metaprogramming}

Metaprogramming is a huge feature of Julia. The key idea is that every
statement in Julia is of the type \texttt{Expression}. Julia operators
by building an Abstract Syntax Tree (AST) from the Expressions. You've
already been exposed to this a little bit: a \texttt{Symbol} (like
\texttt{:PhysicalSciences} is not a string because it is part of the
AST, and thus is part of the parsing/expression structure. One
interesting thing is that symbol comparisons are O(1) while string
comparisons, like always, are O(n)) is part of this, and macros (the
weird functions with an \texttt{@}) are functions on expressions.

Thus you can think of metaprogramming as ``code which takes in code and
outputs code''. One basic example is the \texttt{@time} macro:

    \begin{Verbatim}[commandchars=\\\{\}]
{\color{incolor}In [{\color{incolor}85}]:} macro my\PYZus{}time(ex)
           return quote
             local t0 = time()
             local val = \PYZdl{}ex
             local t1 = time()
             println("elapsed time: ", t1-t0, " seconds")
             val
           end
         end
\end{Verbatim}

    \begin{Verbatim}[commandchars=\\\{\}]

        LoadError: error in method definition: function Base.@time must be explicitly imported to be extended
    while loading In[85], in expression starting on line 1

        

    \end{Verbatim}

    This takes in an expression \texttt{ex}, gets the time before and after
evaluation, and prints the elapsed time between (the real time macro
also calculates the allocations as seen earlier). Note that
\texttt{\$ex} ``interpolates'' the expression into the macro. Going into
detail on metaprogramming is a large step from standard scripting and
will be a later session.

Why macros? One reason is because it lets you define any syntax you
want. Since it operates on the expressions themselves, as long as you
know how to parse the expression into working code, you can ``choose any
syntax'' to be your syntax. A case study will be shown later. Another
reason is because these are done at ``parse time'' and those are only
called once (before the function compilation).


    % Add a bibliography block to the postdoc
    
    
    
    \end{document}
